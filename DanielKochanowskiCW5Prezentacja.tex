\documentclass[]{beamer}
%\usepackage[MeX]{polski}
%\usepackage[cp1250]{inputenc}
\usepackage{polski}
\usepackage[utf8]{inputenc}
\beamersetaveragebackground{blue!10}
\usetheme{Warsaw}
\usecolortheme[rgb={0.1,0.5,0.7}]{structure}
\usepackage{beamerthemesplit}
\usepackage{multirow}
\usepackage{multicol}
\usepackage{array}
\usepackage{graphicx}
\usepackage{enumerate}
\usepackage{amsmath} %pakiet matematyczny
\usepackage{amssymb} %pakiet dodatkowych symboli

\title{Nadciśnienie naczyniowonerkowe}
\author{Daniel Kochanowski}
\institute{UWM}
\date{\today}

\begin{document}

\frame
{
\maketitle
}
\section{Opis dolegliwości}
\begin{frame}{Dolegliwość}

Nadciśnienie naczyniowonerkowe jest to rodzaj nadciśnienia tętniczego spowodowanego przez niedokrwienie nerki i~nadmierną aktywację układu renina–angiotensyna–aldosteron (RAA). Jest najczęstszą postacią nadciśnienia tętniczego wtórnego o~potencjalnie odwracalnej przyczynie, odpowiadając za około 1-2\% wszystkich przypadków nadciśnienia tętniczego.~\cite{1}
\end{frame}


\section{Etiologia}
\begin{frame}{Przyczyny powstawania dolegliwości}
Najczęstsze przyczyny nadciśnienia naczyniowonerkowego to:
\begin{itemize}
\item<2-10> miażdżyca,
\item<3-10> dysplazja włóknisto-mięśniowa,
\item<4-10> choroby zapalne tętnic,
\item<5-10> tętniak,
\item<6-10> zator,
\item<7-10> przetoka tętniczo-żylna,
\item<8-10> jatrogenne,
\item<9-10> torbiel nerki,
\item<10> wrodzona hipoplazja nerki.
\end{itemize}
\end{frame}


\section{Objawy i badania}
\begin{frame}{Objawy}
Nadciśnienie naczyniowo-nerkowe najczęściej rozpoznaje się, kiedy pierwsze epizody nadciśnienia wystąpią przed 30. rokiem życia (głównie u kobiet) lub po 50 (zwłaszcza u mężczyzn), a także, gdy nagle wcześniej dobrze leczone nadciśnienie staje się kłopotliwe do opanowania.
\end{frame}

\begin{frame}{Badania}
\begin{itemize}
\item<2-3>Badania laboratoryjne
\begin{itemize}
\item hipokaliemia,
\item białkomocz,
\item hiperkreatynemia,
\item zwiększona aktywność reninowa osocza.
\end{itemize}
\item<3>Badania obrazowe
\begin{itemize}
\item widoczne zwężenie w dotętniczej angiografii subtrakcyjnej, w USG duplex, w angio-TK lub angio-MR.
\end{itemize}
\end{itemize}
\end{frame}


\section{Leczenie}
\begin{frame}{Leczenie}
Leczenie farmakologiczne:
\begin{itemize}
\item inhibitory konwertazy angiotensyny, antagonisty receptora angiotensyny II,
\item antagonisty kanału wapniowego,
\item leki beta-adrenolityczne.
\end{itemize}
U pacjentów z obustronnym zwężeniem tętnic nerkowych albo ze zwężeniem tętnicy jedynej nerki stosowanie ACEI i sartanów jest przeciwwskazane.
\end{frame}
\begin{frame}{Leczenie cd.}
Leczenie inwazyjne:
\begin{itemize}
\item przezskórna angioplastyka balonowa,
\item przezskórna angioplastyka połączona ze stentowaniem,
\item korekcja chirurgiczna zwężenia.
\end{itemize}
\end{frame}

\section{Bibliografia}
\begin{thebibliography}{9}
\bibitem{1} A.R.~Zeina, W.~Vladimir ,E.~Barmeir : \textit{Fibromuscular dysplasia in an accessory renal artery causing renovascular hypertension: a case report,}
Journal of Medical Case Reports,
Vol.~\textbf{1}, No.~58, (2007).
\end{thebibliography}
\end{document}